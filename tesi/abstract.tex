\begin{abstract}
In questo documento andremo ad analizzare vari esempi, presenti in letteratura, di Attacchi effettuati a sistemi di Malware Detection per Android tramite tecniche di Adversarial Machine Learning. Vedremo le principali metodologie, sia quelle tradizionali che quelle baste su Machine Learning, per l'individuazione di Malware (Malware Detection System) evidenziandone i punti di forza e le debolezze. Esamineremo il concetto di Adversarial Machine Learning con le tipologie di attacco più comuni. Introdurremo il concetto di codice Malevolo (Malware) con alcuni cenni di storia, le varie forme che può assumere ed alcune delle principali tecniche per la sua generazione. Parleremo anche del sistema operativo Android descrivendone l'architettura con i vari livelli che la compongono, la storia con le varie caratteristiche che lo hanno reso celebre e i suoi software principali: le App.
\end{abstract}

\vfill

\epigraph{A year spent in artificial intelligence is enough to make one believe in God.}{\textit{Alan Perlis}}

\vfill

%% vecchio abstract
% In questo documento andremo ad analizzare nel dettaglio il sistema operativo Android, esaminando la sua architettura con i vari livelli che la formano, la storia con le varie caratteristiche che lo hanno reso ampiamente utilizzato ed i suoi software principali: le App. Introdurremo il concetto di Malware con alcuni accenni di storia, le varie tipologie ed alcune tecniche più utilizzate per la creazione di codici malevoli. Parleremo di come, tramite Machine Learning, si possono migliorare o implementare nuove tecniche e sistemi per l'individuazione di Malware (Malware Detection System). Vedremo, infine, come questi sono estremamente efficaci ma altrettanto deboli e vulnerabili contro particolari attacchi appositamente studiati e realizzati tramite tecniche di Adversarial Machine Learning. Includeremo anche una descrizione di alcuni attacchi che utilizzano questa tecnologia presenti in letteratura e le principali (e le più comuni) tecniche di attacco a modelli basati su Machine Learning.