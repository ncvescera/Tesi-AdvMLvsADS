\subsection{Android Malware Detection con Machine Learning}
Di solito, per il nostro problema, vengono impiegati algoritmi di Classificazione \cite{8672711} per distinguere un malware da un normale software. Il Classificatore prende in input un file (\textit{sample}) e lo trasforma in un \textit{feature vector} che utilizzerà poi per effettuare la classificazione. Ci sono due modi per rappresentare un software come \textit{feature vector}:

\begin{enumerate}
    \item \textbf{Modelli Dinamici}: eseguono il codice, osservano il suo comportamento e ne estraggono le informazioni
    \item \textbf{Modelli Statici}: analizzano il software trattandolo come un file binario. Questi sono nettamente più impiegati e studiati rispetto ai Modelli Dinamici.
\end{enumerate}
\ \\
Una volta estratte queste informazioni (\textit{features}), ogni \textit{sample} può essere rappresentato come un vettore \(x \in X\) che ha come classe \(y \in Y\) (per esempio \(0\) per un software benigno e \(1\) per uno maligno) ed utilizzato nei processi di \textit{training} (per derivare la funzione \(f : X \to Y\)) e di \textit{testing}.\\
\\
I Classificatori che rappresentano lo stato dell'arte sono:

\begin{itemize}
    \item Drebin
    \item Stormdroid
    \item MaMaDroid
\end{itemize}

% per allungare potrei aggiungere alcune informazioni su questi, magari con delle subsubsection