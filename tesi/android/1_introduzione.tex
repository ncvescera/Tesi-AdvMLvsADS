\subsection{Introduzione}
Il sistema operativo \textit{Android} fu introdotto in un periodo dove il mercato era colmo di valide alternative ampiamente utilizzate: Nokia con \textit{Symbian}, Apple con \textit{iOS}, BlackBarry con \textit{BlackBarry OS} e Microsoft con \textit{Windows Mobile}. Cosa lo ha reso quindi il Sistema Operativo più adottato (al mondo) \cite{androidwikipedia} su smartphone, dal 2011, e su tablet dal 2013 ? La \textit{libertà}. Android è un software libero (\textit{open source}), a differenza dei suoi rivali, e ciò permette alle case produttrici di utilizzarlo e modificarlo senza dover pagare \textit{royalitis} \cite{androides}. Si basa su \textit{kernel Linux} \cite{androiden} rendendolo facilmente adattabile a moltissimi dispositivi ed hardware differenti. Ovviamente non è l'unico sistema ad essere \textit{open source} e basato su \textit{Linux}, un esempio è \textit{Ubuntu Mobile} che non è riuscito a riscuotere lo stesso successo per via della sua \textit{User Interface} e semplicità di utilizzo. Quest'ultima caratteristica, insieme alla potente SDK con una documentazione molto dettagliata, hanno nettamente influito sulla sua diffusione.
