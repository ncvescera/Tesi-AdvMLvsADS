\subsection{Android e Kotlin}   %% BAIOLETTI APPROVED
\textit{Kotlin} è un linguaggio di programmazione \textit{cross-platform}, \textit{statically typed}, \textit{general-purpose}, creato da \textit{JetBrains} nel 2011. La sua caratteristica principale è l'interoperabilità con Java e l'abilità di generare \textit{bytecode} per le JVM. Ha una sintassi molto concisa e più snella rispetto a Java.

Queste proprietà lo hanno portato, nel 2019 \cite{kotlin}, ad essere etichettato da Google come il suo linguaggio di sviluppo preferito per Android.\\
Ovviamente, tutte le informazioni scritte nelle sezioni precedenti valgono anche per Kotlin.\\
\\
Di seguito un confronto tra \textit{Java} e \textit{Kotlin}. I due codici hanno come obiettivo quello di definire una classe per rappresentare alcuni dati di una persona:\\

\begin{lstlisting}[language=Java, caption=Esempio di codice Java]
public final class Person {
    private String name;
    private int age;
    private float height;
    
    public Person(String name, int age, float height) {
        this.name = name;
        this.age = age;
        this.height = height;
    }
    
    public Person(String name, int age) {
        this.name = name;
        this.age = age;
        this.height = 1.8f;
    }
    
    public String getName() {
        return name;
    }
    
    public void setName(String name) {
        this.name = name;
    }
    
    public int getAge() {
        return age;
    }
    
    public void setAge(int age) {
        this.age = age;
    }
    
    ...
    
    @Override
    public String toString() {
        return "Persona{" +
                "name='" + name + '\'' +
                ", age=" + age  +
                ", height=" + height +
                "}";
    }
    
    ...
}
\end{lstlisting} 

\begin{lstlisting}[language=Java, caption=Esempio di codice Kotlin]
data class Person(
    var name: String, 
    var age: Int, 
    var height: Float = 1.8f
)
\end{lstlisting}

