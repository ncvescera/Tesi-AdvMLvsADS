\subsection{Struttura di un Progetto Android}
\textit{Android Studio} è l'IDE che Google fornisce per la creazione di app per Android. Ogni progetto ha una struttura di cartelle e file ben precisa e complessa che, tramite Android Studio, viene semplificata e mostrata all'utente raggruppando i vari elementi in \textit{moduli} \cite{androidstudio}. I principali moduli sono:

\begin{itemize}
    \item \textbf{manifest}: qui è presente il file \textit{AndroidManifest.xml} al cui interno sono presenti dati importanti come il nome di ogni Activity, il nome del progetto, la dichiarazione di vari Intent, gli elementi hardware necessari al funzionamento, i vari permessi a cui l'app deve avere accesso, ecc.
    \item \textbf{java}: questo modulo contiene tutti i file Java raggruppati nei vari \textit{package} (se presenti).
    \item \textbf{res}: in questa parte sono contenuti tutti i file che non presentano codice al proprio interno: layout XML, immagini, set di colori, icone e altri elementi simili.
\end{itemize}

\begin{lstlisting}[language=XML, caption=Esempio di AndroidManifest.xml]
<?xml version="1.0" encoding="utf-8"?>
<manifest xmlns:android="http://schemas.android.com/apk/res/android"
    package="com.example.toggletest"
    android:versionCode="1"
    android:versionName="1.0" >

    <application
        android:allowBackup="true"
        android:icon="@drawable/ic_launcher"
        android:label="@string/app_name"
        android:theme="@style/AppTheme" >
        <activity
            android:name="com.example.toggletest.MainActivity"
            android:label="@string/app_name" >
            <intent-filter>
                <action android:name="android.intent.action.MAIN" />

                <category android:name="android.intent.category.LAUNCHER" />
            </intent-filter>
        </activity>
    </application>
    
    <uses-permission android:name="android.permission.ACCESS_NETWORK_STATE" />
    <uses-permission android:name="android.permission.ACCESS_WIFI_STATE" />

</manifest>
\end{lstlisting}

\begin{lstlisting}[language=XML, caption=Esempio di Layout XML]
<?xml version="1.0" encoding="utf-8"?>
<LinearLayout xmlns:android="http://schemas.android.com/apk/res/android"
              android:layout_width="match_parent"
              android:layout_height="match_parent"
              android:orientation="vertical" >
    <TextView android:id="@+id/text"
              android:layout_width="wrap_content"
              android:layout_height="wrap_content"
              android:text="Hello, I am a TextView" />
    <Button android:id="@+id/button"
            android:layout_width="wrap_content"
            android:layout_height="wrap_content"
            android:text="Hello, I am a Button" />
</LinearLayout>
\end{lstlisting}

\begin{lstlisting}[language=Java, caption=Esempio di codice di un'Activity scritto in Java]
package com.example.myproject

import android.widget.TextView

public class MainActivity extends Activity {
    private TextView myTextView;
    
    @Override
    public void onCreate(Bundle savedInstanceState) {
        super.onCreate(savedInstanceState);
        setContentView(R.layout.activity_main);
        
        myTextView = findViewById(R.id.mytexview);
        
        myTextView.setText("Hello World !");
    }
}
\end{lstlisting}