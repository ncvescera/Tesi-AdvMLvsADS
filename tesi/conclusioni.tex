\section{Conclusioni}
In questo documento abbiamo introdotto il Sistema Operativo \textit{Android} analizzandone l'architettura con una breve analisi dei vari livelli di cui è formata con un approfondimento sulle JVM utilizzate, la storia, le sue componenti principali, le applicazioni e la loro struttura. È stato introdotto il concetto di \textit{Malware} con un elenco delle varie tipologie di software malevoli esistenti, alcuni accenni di storia ed alcune tecniche per generarli e per la loro individuazione e neutralizzazione. Abbiamo visto alcuni concetti base dell'\textit{Adversarial Machine Learning} con i principali metodi di attacco a modelli di Classificazione per poi arrivare ad applicazioni pratiche contro \textit{Malware Detection System} presenti in letteratura. Da ciò possiamo evincere che i sistemi di Malware Detection basati su intelligenza artificiali sono sì molto potenti e sicuramente più capaci dei tradizionali strumenti per la rilevazione di codici maligni, ma risultano anche vulnerabili ed estremamente insicuri contro \textit{Adversarial Sample} studiati ed appositamente generati per ingannarli.

%% vecchia frase
% ma sono altrettanto vulnerabili 