\subsection{Definizione di Malware}
Il termine \say{Malware} deriva dall'unione di due parole \cite{malware}: \say{malicious} e \say{software} (\textit{software malevolo}) e sta ad indicare qualunque tipo di software indesiderato. Un'altra definizione di Malware, creata da G. McGraw e G. Morrisett \cite{malware} è: \say{qualunque pezzo di codice aggiunto, modificato o rimosso da un software, in modo da danneggiare o alterare il funzionamento del sistema}.\\
\\
Ogni malware ha le seguenti caratteristiche:

\begin{itemize}
    \item \textbf{Riproduzione}: questa è la caratteristica più importante per un malware, dato che gli permette di riprodursi e continuare la sua vita. In alcuni casi, l'eccessiva riproduzione arriva a saturare le risorse della macchina (come RAM e Hard Disk).
    
    \item \textbf{Invisibilità}: tramite questa proprietà i malware riescono a sfuggire ai sistemi di individuazione delle minacce. Può essere realizzata tramite tecniche \textit{polimorfiche} o \textit{metamorfiche}. %% non ho idea di cosa sia, forse il pdf ne parla dopo 
    
    \item \textbf{Propagazione}: un malware deve essere in grado di propagarsi da un dispositivo ad un altro per cercare di infettare più macchine possibili. Un modo molto comune che hanno di spostarsi da un dispositivo infetto ad un altro è tramite la rete (locale o globale) o tramite filesystem locali o remoti.
    
    \item \textbf{Auto-esecuzione}: una volta raggiunto il sistema bersaglio, un malware deve essere in grado di auto avviarsi per poter dare inizio al suo ciclo vitale.
    
    \item \textbf{Corruzione del sistema}: un software di questo tipo può anche andare a minare l'integrità, l'accessibilità e la riservatezza dei dati presenti nel dispositivo attaccato.
\end{itemize}

