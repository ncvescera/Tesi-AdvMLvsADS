\subsection{Creazione di un Malware}
Per creare un Malware si possono utilizzare varie tecniche \cite{malware}, da quelle più semplici che consistono nell'inserimento di una speciale parte di codice all'interno di un programma, fino a quelle più complesse che utilizzano algoritmi sofisticati per generare minacce di tipo \textit{Obfuscated} o \textit{Polymorphic}.\\
\\
Di seguito le andremo ad analizzare brevemente.

\subsubsection{Metodi Basilari}
Come detto prima, queste tecniche consistono nell'inserimento ti parti di codice malevolo all'interno di un altro software. I Malware creati in questo modo possono essere facilmente individuati estraendo dal codice alcune caratteristiche univoche che prendono il nome di \say{signature}.\\

\subsubsection{Polymorphic}
Questo metodo consiste nel riuscire a creare programmi che riescono a mutare, ad ogni nuova infezione, la sintassi della parte di codice malevola, lasciandone inalterata la semantica. La crittografia è la tecnica più utilizzata per generare questo tipo di minacce.

\subsubsection{Obfuscation}
Infine, i Malware prodotti con questa tecnica riescono a trasformare la forma del codice in modo da lasciare inalterate le sue funzionalità ma rendendolo molto complesso da leggere e interpretare.
Alcuni modi per realizzare la Obfuscation sono: 

\begin{itemize}
    \item \textbf{Dead-code}: inserimenti di codice totalmente inutile per rendere difficile e incomprensibile la lettura
    
    \item \textbf{Code Trasportation}: aggiunge salti incondizionati al codice, rendendolo molto complesso, ma lasciandone il flusso di esecuzione originale invariato
    
    \item \textbf{Register Renaming} 
    \item \textbf{Toolkit Paradigm}
\end{itemize}