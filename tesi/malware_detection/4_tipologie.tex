\subsection{Tipologie di Malware}
Esistono svariati metodi per la classificazione dei Malware, il cui scopo è quello di aiutare il tracciamento della paternità, correlazione e identificare la nascita di nuove varianti. In questo documento non utilizzeremo tali classificazioni ma faremo la distinzione tra malware che utilizzano la \textit{Rete Internet} come mezzo di propagazione o di esecuzione e quelli che non la utilizzano. Chiameremo i primi \textit{Network-based Malware}\cite{malware} e gli altri \textit{Ordinary Malware}\cite{malware}.

\subsubsection{Network-based Malware}

\textbf{Spyware}\\
Sono software installati in segreto nella macchina di un utente utilizzati per raccogliere dati a sua insaputa. Anche aziende come Microsoft e Google utilizzano spyware per raccogliere informazioni in segreto\cite{malware}.\\
\\
\textbf{Adware}\\
Questo tipo di software apre in automatico pubblicità e annunci senza il consenso dell'utente. Generalmente sono innocui e puntano solo a far guadagnare il proprietario tramite la pubblicità (che molto spesso si presenta come un pop-up che interrompe l'attività dell'utente), ma ne esistono alcune versioni contenenti spyware, come keylogger, che minacciano la privacy della vittima.\\ 
\\
\textbf{Cookies}\\
I \textit{Cookies} sono informazioni che il browser salva nella macchina dell'utente per vari motivi: salvare le preferenze per i vari siti, mantenere dati per le sessioni server-based, autenticare l'utente, ecc. Di norma questi non sono dannosi e soprattutto non possono essere eseguiti, ma potrebbero rimanere per sempre nel computer della vittima. Esistono dei Malware che rubano e utilizzano le informazioni presenti all'interno di questi Cookies.\\
\\
\textbf{Backdoors}\\
Le \textit{Backdoors}, anche chiamate \textit{Trap Doors}, sono del codice malevolo inserito all'interno di un'applicazione o un sistema operativo che garantiscono all'attaccante l'accesso al sistema della vittima senza dover passare per i consueti metodi di autenticazione.\\
\\
\textbf{Trojan Horse}\\
All'apparenza sembra software utile, ma in realtà il suo scopo è quello di rubare informazioni dell'utente o corrompere i dati.\\
\\
\textbf{Sniffers}\\
Gli \textit{Sniffer} sono programmi che riescono ad intercettare e salvare il traffico che avviene all'interno di una rete. Catturano ogni pacchetto trasmesso o ricevuto, ne analizzano il contenuto e ricavano ogni dato e informazione sensibili presenti al loro interno. Generalmente l'utilizzo degli Sniffer è il primo passo di un \textit{Intrusion Attack}.\\
\\
\textbf{Spam}\\
È uno speciale software che invia a numerosi utenti email contenenti lo stesso messaggio con lo scopo di intasare la casella di posta della vittima, propinare truffe di vario tipo e, a volte, favorire la propagazione di altri malware.\\
\\
\textbf{Botnet}\\
Le \textit{Botnet} sono un insieme di computer infettati (contengono un software chiamato \say{bot}) che possono essere controllati da remoto dall'attaccante. Generalmente vengono utilizzate per performare attacchi di tipo DoS (\textit{Denial of Service}).

\pagebreak

\subsubsection{Ordinary Malware}

\textbf{Virus}\\
I \textit{Virus} sono software in grado di riprodursi da soli, durante la fase di infezione, ed annidarsi in altre applicazioni o documenti. Hanno lo scopo di danneggiare il dispositivo che li ospita (possono arrivare fino alla distruzione dell'intero sistema) e di infettare le periferiche di archiviazione con cui entra in contatto. Un virus viene inserito all'interno di un file tramite tre tecniche:

\begin{enumerate}
    \item \textbf{Pre-pending}
    \item \textbf{Embedding}
    \item \textbf{Post-pending}
\end{enumerate}

Un esempio di ciò è la modifica del file \textit{Autorun.inf}. Questo file risiede in ogni periferica di memoria che contiene musica e serve a riprodurla in automatico quando questa viene collegata al computer. Quando la periferica viene inserita, il sistema operativo va a cercare il file Autorun.inf e lo esegue. Aggiungendo virus all'intero di questo file, l'infezione è garantita. Questo attacco può essere individuato tramite un sistema \say{signature-based} (se le signature sono fornite).\\
\\
\textbf{Worm}\\
Questi software sono in grado di auto-replicarsi, cioè, creare molteplici copie di se stessi e nasconderle all'interno del computer della vittima. Gli \textit{AV Scanner}, sfruttando questa caratteristica, possono riuscire ad individuarli: se sono presenti numerosi file dalle caratteristiche molti simili, allora questo è sintomo di un infezione da worm.\\
\\
\textbf{Logic Bomb}\\
I malware di tipo \textit{Logic Bomb} sono programmi che rimangono inattivi fino quando una determinata condizione non si verifica. Solitamente è la data e l'ora. Questi codici la controllano periodicamente, tramite il sistema operativo, per capire se è il momento di attivarsi. Quando il dato evento sopraggiunge, riescono ad eseguire il proprio codice in automatico.