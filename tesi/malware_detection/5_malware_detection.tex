\subsection{Malware Detection Systems}
Un \textit{Malware Detection System} è un programma in grado di individuare malware presenti in un sistema, qualora ce ne fossero. Una definizione formale di Malware Detection System  è la seguente \cite{malware}:

Un programma di Malware Detection \(D\) è una funzione che lavora su un dominio \(P\) che contiene applicazioni \textit{benigne} e \textit{maligne}. La funzione \(D\) analizza i programmi \(p \in P\) classificandoli in \textit{benigni}, se sono programmi normali oppure \textit{maligni} se contengono un malware. Più in breve:

\[
    D(p) = \begin{cases}
                maligno, & \mbox{se } p \mbox{ contiene codice malevolo} \\ 
                benigno, & \mbox{altrimenti } 
            \end{cases}
\]

La precedente funzione di Malware Detection potrebbe risultare in problemi generando \textit{falsi negativi}, \textit{falsi positivi} o \textit{indecisioni} dipendentemente dalla sua efficienza. Possiamo dunque riscriverla come:

\[
    D(p) = \begin{cases}
                maligno, & \mbox{se } p \mbox{ contiene codice malevolo} \\ 
                benigno, & \mbox{se } p \mbox{ non contiene codice malevolo} \\
                indeciso,& \mbox{se } D \mbox{ non riesce a classificare } p
            \end{cases}
\]

La classe \textit{indeciso} viene assegnata quando non si riesce a capire la natura del programma, generalmente accade quando si incontrano malware nuovi che non sono stati ancora scoperti. Un \textit{falso positivo} è un file innocuo che viene catalogato come malware, infine un \textit{falso negativo} è un malware che viene catalogato come programma benigno.

%% tutto fa brodo, nel casi si rimuove u.u
\subsubsection{Storia dei Sistemi di Malware Detection}
Il primo programma anti-malware fu \say{Flushot Plus} creato da Ross Greenberg nel 1987 \cite{malware}. Il suo scopo era quello di impedire ai vari malware di modificare il contenuto dei file.

Nel 1989, John McAfee rilasciò \say{VirusScan\texttrademark}, un software in grado di individuare ed eliminare diversi virus contemporaneamente.

Successivamente ci fù \say{Wisdom \& Sense (W\&S)}, un \textit{Statistic-based Anomaly Detector} creato al Los Alamos National Laboratory. 

Poi vennero anche \say{Time-based Inductive Machine (TIM)}, \say{Network Security Monito (NSM)}, \say{Information Security Officer's Assistant (IOSA)}.

\pagebreak

\subsubsection{Tecniche e Metodologie}
Generalmente, i produttori di Malware Detection Systems, tengono traccia dei nuovi programmi, li analizzano e li aggiungono in apposite liste \cite{malware}:

\begin{itemize}
    \item \textbf{White List}: qui ci vanno i software innocui
    \item \textbf{Black List}: questo è il posto per i malware
    \item \textbf{Gray List}: qui aggiungono i software che generano \textit{indecisioni}
\end{itemize}

Per questi ultimi, quando un software entra a farvi parte, viene fatto eseguire in un ambiente controllato e isolato per poterlo classificare con più precisione. Qualora uno di questi programmi venisse rilevato come malware, i produttori del sistema rilascerebbero un aggiornamento permettendo così agli utenti di scaricarlo ed avere sempre un database delle minacce aggiornato.\\
\\
Qui di seguito andremo ad analizzare le più comuni tecniche di Malware Detection.\\
\\
\textbf{Signature-based \& Anomaly-based}\\
Tutti i malware scanner utilizzano tecniche \textit{signature-based} e \textit{anomaly-based} per andare ad identificare la natura di un programma. Queste possono essere applicate in maniera \textit{dinamica} o \textit{statica}. La prima necessita di far eseguire il programma e ricava informazioni a \textit{run time}, l'altra estrapola le informazioni senza la necessità di avviare il file. Infine, ne esiste un'altra, chiamata \textit{ibrida}, che utilizza una combinazione del metodo \textit{statico} e di quello \textit{dinamico}.

La tecnica \textit{Signature-based} consiste nel confrontare l'impronta del file, chiamata anche \textit{signature} (firma), con altre impronte di malware presenti in un database. Il suo vantaggio è una buona efficacia, ma ha il grande problema che non riesce ad individuare nuovi malware non ancora presenti all'interno di questi database.

I sistemi che utilizzano la metodologia \textit{Anomaly-based} invece, catalogano i programmi in base al loro comportamento: analizzano le varie azioni che compiono cercando di individuare procedure sospette che trascendono dal normale utilizzo della macchina.\\
\\
\textbf{Heuristic-based}\\
L'\textit{Intelligenza Artificiale} (AI) è stata molto utilizzata insieme alle tecniche Signature-based e Anomaly-based per aumentare la loro efficienza. Vengono utilizzate \textit{Reti Neurali} (NN) soprattutto per la loro caratteristica di adattarsi bene ai cambiamenti e per l'abilità di effettuare predizioni. Anche gli \textit{Algoritmi Genetici} sono impiegati nei problemi di malware detection, per derivare regole di classificazione e per la scelta di features e parametri. Questi, per funzionare, applicano principi della biologia evolutiva quali: \textit{Ereditarietà}, \textit{Mutazioni}, \textit{Selezione} e \textit{Combinazione}.\\
Modelli \textit{Statistici e Matematici} vengono impiegati anche in questo campo, applicandoli ad informazioni del sistema come: connessioni attive, bandwidth, utilizzo della memoria, chiamate di sistema, ecc.

\subsubsection{Tecnologie}

\textbf{Host-based}\\
I sistemi di detection Host-based, monitorano costantemente e dinamicamente lo stato e il comportamento del sistema, controllando se ci sono attività interne o esterne che mettono a rischio l'integrità della macchina. Questi vengono anche chiamati \say{in-the-box} perché risiedono all'interno dello stesso dispositivo che proteggono.
Il vantaggio di questi sistemi è che sono in grado di proteggere l'host, in maniera molto efficacie, dall'interno, ma sono deboli a contrastare attacchi esterni.\\
\\
\textbf{Network-based}\\
Queste tecnologie prendono il nome di \textit{Network-based Intrusion Detection Systems} (IDS) e vengono utilizzate per \say{sniffare} tutti i pacchetti da e per ogni nodo della rete ed analizzarli. Possono esserci moduli di questo tipo per ogni segmento di rete (che monitorano il traffico di quel pezzo) oppure un modulo per ogni nodo. Questi vengono anche chiamati \say{out-of-the-box} dato che risiedono fuori dal dispositivo che proteggono. Il loro vantaggio è di riuscire a prevenire attacchi dall'esterno, ma non riescono a proteggere da attacchi interni.\\
\\
\textbf{Hybrid-based}\\
Un approccio misto tra Host-based e Network-based può essere utilizzato. Ogni nodo della rete ha un particolare modulo che raccoglie informazioni e le manda ad un dispositivo principale che ne effettua l'analisi e la classificazione. L'approccio ibrido riesce a sopperire ai difetti delle singole tecnologie utilizzate.\\
\\
\textbf{Virtual Machines}\\
Le \textit{Virtual Machine} (VM) possono essere utilizzate nell'ambito della malware detection in quanto permettono di avere una copia della macchina reale in grado di essere isolata da quest'ultima e garantiscono un'elevata sicurezza, efficienza di esecuzione ed il pieno controllo delle risorse. Queste servono come ambiente sicuro dove far eseguire programmi per poi analizzarli e catalogarli. Alcune tipologie di macchine virtuali sono:

\begin{itemize}
    \item \textbf{Sandbox}: un ambiente controllato dove le varie risorse sono accessibili solo tramite tramite API fornite dalla VM. Qui sono fatti eseguire i vari software sospetti che vengono monitorati, analizzati ed infine viene stilato un report finale.
    \item \textbf{Emulation}: viene simulato l'intero sistema del computer (\textit{emulation}) per poterci far eseguire un altro sistema operativo (\textit{guest}) che risulta isolato dal sistema originale (\textit{host}).
\end{itemize}
\ \\    %% per andare a capo dopo l'elenco puntato
\textbf{Web-based}\\
Esistono siti web che hanno la capacità di effettuare scansioni dell'intero sistema di un computer, di hard disk, aree del sistema a rischio, cartelle e file. Questa è una buona soluzione per chi non vuole installare anti-malware nel proprio dispositivo dato che questi possono essere disabilitati da alcuni tipi di software malevoli.\\
\\
\textbf{Application Protocol-based}\\
\textit{Application Protocol-based Intrusion Detection System} (APIDS) è uno speciale sistema di monitoraggio che analizza uno specifico protocollo applicativo. Questo sistema è formato da un \textit{agent} posizionato tra diversi gruppi di server che monitora costantemente lo stato di un determinato protocollo applicativo utilizzato da questi. Un esempio è un APIDS posto tra un Web Server ed un Database Management System che analizza il traffico del protocollo SQL.
